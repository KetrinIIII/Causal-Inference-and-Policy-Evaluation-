\documentclass[12pt]{article}
\usepackage{graphicx}
\usepackage{amsmath}
\usepackage{geometry}
\usepackage{float}
\usepackage{float} 
\geometry{a4paper}

% Title Page
\begin{document}

\begin{titlepage}
    \centering
    \vspace*{1.5in}
    
    \Huge
    Università degli Studi di Milano \\
    \vspace{0.5in}
    \LARGE
    \textbf{Data Science for Economics} \\
    \vspace{0.5in}
    \Large
    Causal Inference and Policy Evaluation \\
    \vspace{1.5in}
    \Huge
    \textbf{Colonial Plantations and Their Role in Shaping Modern Economies: Evidence from IV Analysis} \\
    \vfill
    {\Large Ketrin Hristova} \\
    \normalsize Matriculation Number: 13209A \\
    \vspace{0.5in}
    
\end{titlepage}

% Declaration Page
\newpage
\begin{center}
    \textbf{Abstract}
\end{center}

\noindent
This paper explores the long-term economic effects of colonial plantation economies on post-colonial development, with a specific focus on the level of plantation economies during the colonial period and their relationship to GDP per capita in 2012. Using a dataset of 31 former colonies in Asia and Africa, we employ both ordinary least squares (OLS) and two-stage least squares (2SLS) regression models to examine the impact of colonial plantation presence on modern economic outcomes. The study addresses endogeneity concerns by using the extent of plantations during the colonial period as an instrumental variable. Our results reveal a significant positive association between the level of colonial plantation economies and GDP per capita in 2012, indicating that regions with more significant plantation economies during the colonial period experienced higher levels of economic development in the modern era. The findings contribute to understanding the lasting economic legacies of colonialism and provide insights into the factors that shaped contemporary economic disparities between former colonies.
\newpage
\section{Introduction}

\vspace{0.5in}
The legacy of European colonialism continues to shape the economic and social landscapes of many nations today. Over several centuries, European powers established colonial systems that fundamentally altered the institutional, political, and economic structures in colonized regions. However, these colonial experiences were not uniform—different colonies experienced distinct forms of exploitation and varying degrees of economic development. One key factor contributing to this diversity is the nature of the colonial economy, with plantation-based economies playing a particularly significant role in shaping the long-term economic outcomes of affected regions.

\noindent Plantation-based colonial economies, where the colonizers exploited local human and natural resources for agricultural production, particularly in the form of cash crops, were widespread across parts of Africa and Asia. These economies were marked by high levels of exploitation, which not only created deep economic and social inequalities but also influenced the trajectories of post-colonial development. Given the extractive nature of these plantation economies, it is plausible to argue that regions with a higher dependence on plantations during the colonial period might exhibit different economic outcomes in the modern era.

\vspace{0.2in}
\noindent This paper builds on the growing body of literature examining the enduring economic consequences of colonialism. A foundational contribution to this field is the work of Acemoglu, Johnson, and Robinson (2001), who argue that the colonial institutions set up by European powers, including plantation-based economies, significantly influenced the long-term economic development of former colonies. According to their theory, the extractive colonial institutions that were established in colonies with valuable natural resources, such as plantations, left a lasting legacy of poor economic performance. These extractive institutions tended to suppress inclusive economic policies, which in turn hindered growth and development post-independence.

\noindent Further, Ziltener, Künzler, and Walter (2020) provide a more nuanced understanding of colonial impacts, presenting a comprehensive dataset that measures various facets of colonial history across Africa and Asia. Their work underscores how different colonial regimes created divergent economic and institutional legacies, with plantation economies often linked to higher levels of exploitation and inequality. These colonial economic systems not only shaped the immediate economic environment but also had profound, long-term effects on the ability of post-colonial states to develop institutions that promote sustained economic growth.

\vspace{0.2in}
\noindent We investigate the lasting impact of plantation economies on the economic development of former colonies, focusing on the relationship between the level of plantation economies during colonialism and GDP per capita in 2012. By using a dataset of 31 former colonies in Africa and Asia, we employ a 2-stage least squares (2SLS) regression models to address potential endogeneity concerns, with the importance of plantation economies instrumented by the extent of plantations and rainfall levels. We hypothesize that regions with a greater reliance on plantations during colonization experienced higher levels of economic development in the modern period, as measured by GDP per capita in 2012.

\noindent Our findings reveal a statistically significant positive relationship between the extent of plantation economies during the colonial period and GDP per capita in 2012, suggesting that the economic structures established during colonization have had lasting effects on post-colonial economic performance. 
The following sections of the paper will provide historical context, review the relevant literature, discuss the empirical model used, present the results, and conclude with the implications of our findings.

\vspace{1 in}
\section{Historical Background}
\vspace{0.2in}
The expansion of European colonial empires from the late 15th century onwards significantly reshaped the economic and political landscapes of regions across Africa and Asia. What began as voyages of exploration by figures like Vasco da Gama and Christopher Columbus rapidly turned into a global pursuit for economic dominance, driven primarily by the quest for valuable natural resources, new trade routes, and agricultural products. By the 16th century, European powers, including the Portuguese, Spanish, British, French, and Dutch, had established colonies in the Americas, Asia, and Africa, each with its own form of colonial relationship and exploitation. The plantation economy, which became particularly prominent in tropical regions, was one of the most defining features of European colonialism (Acemoglu et al., 2001; Lange, 2006).

\vspace{0.2in}
\noindent The plantation-based colonial economies were established with the purpose of cultivating highly profitable cash crops such as sugar, tobacco, cotton, coffee, and rubber. These crops were integral to the European economy, fueling the growth of European industries and trade. Plantations were predominantly found in regions with tropical climates, such as Southeast Asia, parts of sub-Saharan Africa, and the Caribbean. In these regions, the European colonizers implemented highly exploitative systems of labor, relying on slavery, indentured servitude, and coerced labor from indigenous populations. The brutality of these systems, which included harsh working conditions, lack of basic rights, and a complete disregard for the welfare of workers, resulted in long-lasting social and economic inequalities (Ziltener et al., 2019).

\vspace{0.2in}
\noindent In Southeast Asia, the Dutch and British colonists established large plantations for rubber and other agricultural products, particularly in Indonesia, Malaysia, and Singapore. These plantations were worked by indentured laborers, often from India and China, who were subjected to exploitative contracts that benefited colonial powers at the expense of the laborers. In French Indochina (Vietnam, Laos, and Cambodia), a similar model was employed with plantations growing rice, rubber, and opium, often using both local labor and imported workers (Gordon, 2001). The economic structures created by these plantations relied on extraction and control, with little regard for local welfare or the development of inclusive economic institutions (Fitzpatrick, 1980).

\vspace{0.2in}
\noindent In Africa, the colonization process intensified in the late 19th century, particularly during the "Scramble for Africa" in the 1880s. European powers carved up the continent into colonies, often ignoring existing political and social structures. Colonies like Kenya, Côte d'Ivoire, and Zimbabwe saw the establishment of large agricultural plantations for crops such as coffee, cocoa, and tobacco. The forced labor systems established by the Belgian, British, French, and German colonizers exploited local populations, integrating them into global commodity markets controlled by the colonial powers. In some instances, these systems were directly tied to the economic interests of the colonizers, as seen in the Belgian Congo, where forced labor was used to harvest rubber and other valuable resources.

\vspace{0.2in}
\noindent As these colonies transitioned to independence in the mid-20th century, the economic and social impacts of these extractive plantation economies were deeply felt. The violence of decolonization, including resistance movements and uprisings, was often a direct response to the exploitation and economic structures left behind by colonial powers (Acemoglu et al., 2001). The legacy of plantation economies, marked by inequality, extractive institutions, and reliance on cash crop monocultures, left many newly independent nations with economies that were dependent on a narrow range of products, often without the institutional infrastructure needed for long-term development. These economies were disproportionately dependent on agricultural exports, leading to a lack of diversification, low industrialization, and high vulnerability to global commodity price fluctuations (Ziltener et al., 2019).

\vspace{0.2in}
\noindent The long-term economic impact of colonial plantation economies can be seen in the modern economic performance of many former colonies. The structures created during the colonial era—whether through land tenure systems, labor practices, or commodity-focused economies—often left countries with limited capacity for broad-based economic development. In many cases, this has resulted in slower economic growth and poorer development outcomes in the post-colonial era. The economic performance of many former colonies is still impacted by these colonial legacies, as evidenced by the persistent inequality and reliance on a narrow range of exports (Ziltener et al., 2019; Acemoglu et al., 2001).


\vspace{1 in}
\section{Literature review}

\vspace{0.2 in}
The long-term economic effects of colonialism on African development have been extensively examined through the lens of institutional quality and its persistence over time. Acemoglu, Johnson, and Robinson (2001) laid a foundational framework in The Colonial Origins of Comparative Development, arguing that the institutions established during the colonial era had lasting implications for post-colonial economic outcomes. According to their theory, colonies that experienced extractive institutions—often exemplified by plantation economies—suffered from long-term economic stagnation due to the concentration of power and resources in the hands of a few. In contrast, regions with more inclusive institutions saw greater economic development and more equitable growth.

\vspace{0.2in}
\noindent Gareth Austin's work, African Economic Development and Colonial Legacies, expands on this theory, emphasizing the complex interplay between colonial economic systems, resource extraction, and post-independence economic performance. Austin notes that the legacies of colonial rule, particularly those based on extractive industries such as plantations, shaped African economies by reinforcing patterns of inequality and limiting economic diversification. Plantation economies, which relied heavily on forced and cheap labor, created deeply entrenched systems of economic exploitation. These systems left behind a legacy of unequal development, where regions with a heavy dependence on plantations during the colonial period often found it difficult to transition to more diversified and sustainable economies after independence. Austin suggests that the wealth generated from plantation economies was disproportionately funneled to colonial powers, rather than reinvested into the local economies, further exacerbating economic disparities post-independence.

\vspace{0.2in}
\noindent Nathan Nunn (2008) in Understanding the Long-Run Effects of Africa’s Slave Trades and Engerman and Sokoloff (1997) in Factor Endowments, Inequality, and Paths of Development Among New World Economies support these claims by showing that the extractive nature of colonial economies—particularly those focused on cash crops like sugar, tobacco, and cotton—created significant economic challenges for former colonies. These systems hindered the development of inclusive economic institutions and resulted in long-term underdevelopment. Nunn specifically highlights the detrimental effect of the African slave trade, which decimated populations and weakened social structures, leaving lasting scars on the economic landscapes of the affected regions.

\vspace{0.2in}
\noindent Furthermore, Ziltener, Künzler, and Walter (2020) in Measuring the Impacts of Colonialism: A New Data Set for the Countries of Africa and Asia provide empirical evidence linking colonial institutions to contemporary economic outcomes. Their dataset, which includes both historical colonial data and modern economic indicators, allows for a detailed analysis of how colonial exploitation, particularly in plantation economies, shaped the trajectory of economic development in former colonies.

\vspace{0.2in}
\noindent The role of plantation-based colonial economies in shaping post-colonial economic outcomes is also explored through the framework of "extractive" and "inclusive" institutions, as outlined by Acemoglu et al. (2001). In their view, extractive institutions concentrated power and economic resources in the hands of a few, often undermining the capacity of local populations to invest in their own economic futures. These extractive systems were particularly prominent in plantation economies, where wealth extraction from forced labor systems created deep-seated inequalities that persisted long after independence.

\noindent Furthermore, the persistence of colonial plantation economies as a determinant of modern economic outcomes has been widely discussed in institutional economics and development literature. Theoretical frameworks such as Acemoglu et al. (2001) persistence of institutions hypothesis suggest that extractive colonial institutions, including plantation economies, have long-term effects on economic performance. The plantation system often fostered coercive labor practices, reinforced wealth concentration, and limited human capital development, leading to persistent inequalities that influence contemporary economic trajectories.

\vspace{0.2in}
\noindent This body of research underscores the crucial role of colonial economic systems, especially plantation-based ones, in determining the economic trajectories of former colonies. In this project, we aim to build on this literature by investigating the specific impact of plantation economies on GDP in 2012 across former African and Asian colonies, with a focus on how these colonial legacies continue to shape economic outcomes in the post-colonial period.

\vspace{0.8 in}

\section{Data}

\vspace{0.2 in}
The data used in this study are compiled from multiple sources, each previously utilized in the literature on colonial institutions and economic development, with variables selected based on their theoretical relevance to the mechanisms through which colonial plantations influenced modern economies.

\vspace{0.2 in}
\noindent The first dataset originates from Assenova and Regele (2017), which builds upon Acemoglu, Johnson, and Robinson’s (2001, 2012) framework of institutional persistence, focusing on how different colonial strategies, particularly the extraction of resources and labor, shaped long-term economic trajectories. From this dataset, the study extracts key variables, including \textit{logpgp12}, the log of GDP per capita (PPP) in 2012 sourced from the World Bank, which serves as the dependent variable to capture the modern economic outcome influenced by historical plantation structures; \textit{f\_french}, a binary variable indicating whether a country was a French colony, included because previous studies suggest that French colonial rule differed from British rule in its approach to governance, legal systems, and economic structures, making this an essential control variable; \textit{landlock}, a dummy variable indicating whether a country is landlocked, a crucial control due to historical structural disadvantages faced by landlocked countries in trade and economic integration (faye2004); \textit{iron}, a binary indicator for the presence of iron resources, included to account for natural resource endowments that could confound the relationship between plantations and economic development; and \textit{latabs}, absolute latitude, which serves as a geographical determinant of economic development based on the McArthur and Sachs (2001) framework, positing that latitude is correlated with factors such as climate, disease environment, and agricultural productivity.

\vspace{0.2 in} 
\noindent The second dataset utilized is the \textit{Correlates of War (COW) Dataset}, which captures all combat-related military deaths in conflicts from 1816 to 2007, acknowledging that colonial violence and conflict are key determinants of economic divergence across post-colonial states. The main variable extracted from this dataset is \textit{deaths\_per\_pop}, which measures the number of combat-related deaths per capita during decolonization periods and is included to control for the destabilizing effects of violent independence struggles on subsequent economic development.

\vspace{0.2 in} 
\noindent The third dataset is drawn from Ziltener, Kunzler, and Walter’s (2017) work, \textit{Measuring the Impacts of Colonialism}, which provides granular data on colonial economic structures, including plantations, as plantation economies have been identified in the literature as a key mechanism through which colonial extractive institutions persisted. This dataset contributes the variable \textit{level\_of\_plantation}, which measures the extent of plantation economies by land occupied and share of total exports derived from plantations; however, for analytical simplicity and to capture the presence or absence of plantation economies, this variable was transformed into a binary dummy variable, \textit{plantation\_dummy}. The dummy variable is coded as 1 if a country had any level of plantation economy (i.e., \textit{level\_of\_plantation} = 1 or 2), and 0 otherwise. This transformation allows the study to focus on the dichotomy of having or not having plantations, simplifying the modeling process while still capturing the fundamental effect of plantation economies on modern economic outcomes. \textit{political\_violence} is a measure of political violence in the post-colonial period and is included as a control for institutional stability; \textit{COLYEARS} represents the number of years a country was under colonial rule, which is a key factor in determining the depth of institutional legacies (Acemoglu, 2001); \textit{power\_transfer\_during\_decolonization} is a variable indicating whether the transition from colonial rule was planned and coordinated or abrupt and unstructured, included due to literature suggesting that unplanned transitions are associated with weaker post-colonial institutions and slower economic growth; and \textit{plantationsduringcolonialperi}, an indicator of the size of plantations during the colonial period, serving as the instrumental variable (IV) for \textit{level\_of\_plantation}, as it is a direct determinant plantation economies, and it does not directly affect \textit{logpgp12}, making it a suitable instrument for identifying the causal relationship between plantation economies during the colonial period and modern economic outcomes.
The use of \textit{plantationsduringcolonialperi} as an instrumental variable is grounded in the idea that the extent of plantation economies during the colonial period is exogenous to modern GDP but highly predictive of plantation size in the colonial period. This aligns with the literature on historical determinism in economic development, which argues that colonial economic structures, particularly those dependent on large-scale plantation agriculture, shaped the long-run institutional and economic outcomes of former colonies.

\newpage

\section{Empirical Model}

\vspace{0.2 in} 
We investigate the impact of a plantation-based colonial economy on contemporary economic outcomes, particularly GDP in 2012. A simple OLS regression framework is initially used, where the output variable is GDP, and the predictor is a plantation economu dummy variable. However, to account for potential endogeneity, we enhance this model by employing an instrumental variable (IV) approach and incorporating relevant control variables. 

\vspace{0.2 in} 
\noindent The main specification of our model utilizes the size of plantations during the colonial period as an instrument for the endogenous plantation economy variable. This approach provides a source of variation in the plantation economy that is exogenous to GDP outcomes, helping mitigate endogeneity issues that may arise from reverse causality or omitted variable bias. Therefore, the validity of our instrument is a critical consideration. The instrument must satisfy both the excludability and relevance conditions. The size of plantations likely satisfies both conditions. 

\vspace{0.2 in} 
\noindent An alternative IV candidate considered was the rainfall levels, specifically the average rainfall depth (in mm) for each country, as has been used in similar studies. For instance, Miguel, Satyanath, and Sergenti (2004) used rainfall variation as an instrument in the context of civil conflict and economic growth, demonstrating a statistically significant relationship between economic growth and conflict incidence in Africa. However, in our analysis, rainfall did not perform well as an instrument, and its inclusion resulted in weaker model outcomes compared to plantation size, which is more consistent with the conditions required for a 2-SLS approach.

\vspace{0.2 in} 
\noindent The relevance condition for our instrument is met through the plantation size variable's significant impact on the existence and scale of plantations in colonial economies. Large plantations were typically situated in territories with adequate land availability, making the size of plantations a meaningful predictor of colonial plantation economies. Empirically, the first-stage regression coefficients confirm this theoretical framework, showing a highly significant positive relationship between plantation size and the plantation economy variable. Moreover, the F-statistic for the first-stage regression, ranging between 10 and 15, indicates that the instrument has strong explanatory power, a crucial feature for ensuring the reliability of the IV estimation. In our context, plantation size is expected to affect modern GDP, but this effect operates solely through the presence and scale of plantation economies established during the colonial period.

\vspace{0.2 in} 
\noindent In order to examine the impact of plantation economies on modern economic outcomes, we employ a two-stage least squares (2SLS) instrumental variables approach. The first-stage model is specified as follows:

\[
\text{PlantationEconomy}_{ic} = \alpha_1 + \beta_1 \text{PlantationSize}_{ic} + \Gamma \mathbf{X}_{ic} + \Delta \mathbf{c} + \mathbf{E}_{ic}
\]

\noindent where \( \text{PlantationEconomy}_{ic} \) is the endogenous variable representing the plantation economy dummy for country \( i \) under colonizer \( c \), \( \text{PlantationSize}_{ic} \) is the instrumental variable (plantation size in country \( i \) colonized by \( c \)), \( \mathbf{X}_{ic} \) represents a vector of control variables, \( \mathbf{c} \) are the colonizer fixed effects, and \( \mathbf{E}_{ic} \) is the error term in the first stage.

\noindent The predicted values from this first-stage regression (\( \hat{\text{PlantationEconomy}}_{ic} \)) are then used in the second stage of the 2SLS estimation. The second-stage model is specified as:

\[
\text{GDP12}_{ic} = \alpha' + \beta' \hat{\text{PlantationEconomy}}_{ic} + \gamma \mathbf{X}_{ic} + \delta \mathbf{c} + \boldsymbol{\epsilon}_{ic}
\]

\noindent where \( \text{GDP12}_{ic} \) is the modern GDP measure for country \( i \) under colonizer \( c \), \( \hat{\text{PlantationEconomy}}_{ic} \) is the predicted plantation economy from the first-stage regression, \( \beta' \) is the coefficient of interest, \( \mathbf{X}_{ic} \) represents a vector of control variables, \( \mathbf{c} \) are colonizer fixed effects, and \( \boldsymbol{\epsilon}_{ic} \) is the error term in the second stage.

\noindent This two-stage method allows us to isolate the causal effect of plantation economies on modern economic outcomes by instrumenting the endogenous plantation economy variable with plantation size. The instrument's relevance is ensured by the statistically significant relationship between plantation size and the plantation economy in the first stage, as well as a strong first-stage F-statistic.

\noindent By using 2SLS, we aim to address potential endogeneity issues and obtain consistent estimates of the causal effect of plantation economies on GDP.

\section{Results}

\vspace{0.5 in} 
\setlength{\pdfpagewidth}{8.5in} 
\setlength{\pdfpageheight}{11in} 

\begin{flushleft}
\resizebox{\textwidth}{!}{
\begin{tabular}{lccccc}
\multicolumn{6}{c}{Table 1: FIRST STAGE RESULTS} \\ \hline
 & (1) & (2) & (3) & (4) & (5) \\
VARIABLES & Main Model & M2: Without IV & M3: Without Fixed Eff & M4: Alt IV Rainfall & M5:Workers1988 Dep  \\ \hline
 &  &  &  &  &  \\
plantation size & 0.464*** & - & 0.464*** &  & 0.464*** \\
 & (0.033) & - & (0.105) &  & (0.033) \\
COLYEARS & -0.002** & - & -0.002 & -0.002 & -0.002** \\
 & (0.000) & - & (0.001) & (0.002) & (0.000) \\
iron & 0.110*** & - & 0.110 & 0.066 & 0.110*** \\
 & (0.017) & - & (0.075) & (0.073) & (0.017) \\
landlock & -0.358 & - & -0.358 & -0.607 & -0.358 \\
 & (0.272) & - & (0.210) & (0.441) & (0.272) \\
Asia & -0.103* & - & -0.103 & 0.051 & -0.103* \\
 & (0.049) & - & (0.245) & (0.072) & (0.049) \\
f\_french & 0.283*** & - & 0.283 & -0.111 & 0.283*** \\
 & (0.051) & - & (0.429) & (0.212) & (0.051) \\
political\_violence & -0.094 & - & -0.094 & -0.083 & -0.094 \\
 & (0.191) & - & (0.188) & (0.329) & (0.191) \\
latabs & -1.907*** & - & -1.907 & -1.750*** & -1.907*** \\
 & (0.365) & - & (1.254) & (0.134) & (0.365) \\
deaths\_per\_pop & 0.033** & - & 0.033 & 0.107*** & 0.033** \\
 & (0.009) & - & (0.071) & (0.018) & (0.009) \\
power trans decol & -0.091 & - & -0.091 & -0.021 & -0.091 \\
 & (0.108) & - & (0.201) & (0.225) & (0.108) \\
immigr foreign w & -0.102 & - & -0.102 & 0.015 & -0.102 \\
 & (0.089) & - & (0.138) & (0.257) & (0.089) \\
Colony, FRA & 0.200** & - & - & 0.192* & 0.200** \\
 & (0.071) & - & - & (0.080) & (0.071) \\
Colony, ITA & 0.664** & - & - & -0.129 & 0.664** \\
 & (0.185) & - & - & (0.717) & (0.185) \\
Colony, NL & 0.584** & - & - & 0.526 & 0.584** \\
 & (0.162) & - & - & (0.577) & (0.162) \\
Colony, POR & 0.803*** & - & - & 0.378 & 0.803*** \\
 & (0.174) & - & - & (0.781) & (0.174) \\
Colony, UK & 0.369*** & - & - & -0.080 & 0.369*** \\
 & (0.076) & - & - & (0.516) & (0.076) \\
rainfall &  &  &  & -0.000 &  \\
 &  &  &  & (0.000) &  \\
Observations & 32 & 32 & 32 & 32 & 32 \\
F-Statistic & 14.41 & - & 10.25 & 0.11 & 10.24 \\
 R-squared & 0.734 & 0.734 & 0.734 & 0.386 & 0.734 \\ \hline
\multicolumn{6}{c}{ Robust standard errors in parentheses} \\
\multicolumn{6}{c}{ *** p$<$0.01, ** p$<$0.05, * p$<$0.1} \\
\end{tabular}
}
\end{flushleft}

\setlength{\pdfpagewidth}{8.5in} 
\setlength{\pdfpageheight}{11in} 

\begin{flushleft}
\resizebox{\textwidth}{!}{
\begin{tabular}{lccccc}
\multicolumn{6}{c}{Table 2: 2SLS RESULTS} \\ \hline
 & (1) & (2) & (3) & (4) & (5) \\
VARIABLES & Main Model & M2: Without IV & M3: Without Fixed Eff & M4: Alt IV Rainfall & M5: Workers 1988 Dep \\ \hline
 &  &  &  &  &  \\
plantation econ. & 0.719*** & 0.295 & 0.643 & 3.196 & 0.460*** \\
 & (0.209) & (0.405) & (0.405) & (3.989) & (0.156) \\
COLYEARS & 0.003*** & 0.002*** & 0.003* & 0.007 & 0.002*** \\
 & (0.000) & (0.000) & (0.001) & (0.009) & (0.000) \\
iron & -0.101** & -0.071 & -0.090 & -0.277 & -0.198*** \\
 & (0.051) & (0.059) & (0.063) & (0.416) & (0.029) \\
landlock & -0.077*** & -0.324 & -0.120 & 1.361 & -0.525*** \\
 & (0.029) & (0.247) & (0.386) & (1.643) & (0.124) \\
Asia & 0.266 & 0.258 & 0.395* & 0.313** & 0.634*** \\
 & (0.188) & (0.333) & (0.210) & (0.134) & (0.099) \\
f\_french & 0.159*** & 0.116* & -0.207 & 0.410*** & 0.096 \\
 & (0.027) & (0.050) & (0.235) & (0.040) & (0.070) \\
political\_violence & 0.481* & 0.453 & 0.368* & 0.645 & -0.013 \\
 & (0.251) & (0.263) & (0.216) & (0.478) & (0.214) \\
latabs & 5.083*** & 4.540* & 4.394*** & 8.256 & 4.238*** \\
 & (1.056) & (2.022) & (1.592) & (5.447) & (0.913) \\
deaths\_per\_pop & -0.013 & 0.028 & 0.014 & -0.253 & 0.057 \\
 & (0.057) & (0.084) & (0.078) & (0.314) & (0.036) \\
power trans decol & 0.259 & 0.261 & 0.082 & 0.244 & 0.026 \\
 & (0.237) & (0.301) & (0.181) & (0.534) & (0.185) \\
immigr foreign w & 0.417* & 0.438 & 0.341** & 0.295 & 0.260 \\
 & (0.236) & (0.291) & (0.158) & (0.738) & (0.210) \\
Colony, FRA & -0.629** & - & - & -1.167*** & 0.074 \\
 & (0.295) & - & - & (0.262) & (0.275) \\
Colony, ITA & 0.156 & 0.699 & - & 0.114 & - \\
 & (0.432) & (0.890) & - & (0.610) & - \\
Colony, NL & -0.112 & 0.680 & - & -1.604 & -0.376** \\
 & (0.204) & (0.572) & - & (2.605) & (0.175) \\
Colony, POR & -0.617*** & 0.121 & - & -1.790 & -1.820*** \\
 & (0.026) & (0.127) & - & (2.146) & (0.034) \\
Colony, UK & - & 0.537* & - & - & - \\
 & - & (0.236) & - & - & - \\
Observations & 31 & 31 & 31 & 31 & 30 \\
 R-squared & 0.653 & 0.694 & 0.646 & - & 0.693 \\ \hline
\multicolumn{6}{c}{ Robust standard errors in parentheses} \\
\multicolumn{6}{c}{ *** p$<$0.01, ** p$<$0.05, * p$<$0.1} \\
\end{tabular}
}
\end{flushleft}

\vspace{0.5 in} 
\noindent Column 1 in Table 1 presents the results from the first-stage regression of our main empirical model. These estimates provide a reliable foundation for understanding the relationship between plantation economies and modern economic outcomes. The first-stage regression confirms a statistically significant and positive coefficient for plantation economies, instrumented by plantation size. This result indicates that our instrumental variable is strongly correlated with the endogenous regressor, validating plantation size as a valid instrument. The positive coefficient of 0.464 suggests that larger plantation economies are associated with greater plantation sizes, capturing exogenous variation in plantation economies. Robust standard errors of 0.033 confirm that this effect is both statistically significant and precise.

\noindent The coefficients on the colonial origin variables—such as those for French, Dutch, Portuguese, and UK colonies—are all positive and statistically significant. This reinforces the theoretical framework that extractive institutions in these colonies were designed to favor large-scale plantation systems. The positive coefficients for former colonial powers suggest that the institutional framework established by these colonial powers had long-term consequences, particularly in the agricultural and plantation sectors.

\noindent Colonization years (\texttt{COLYEARS}) show a negative and statistically significant relationship with plantation size (\texttt{-0.002}). This finding suggests that the longer a country was colonized, the less likely it is to have a large plantation economy in the present day. This could reflect the idea that prolonged colonization led to more diversified economies or different forms of land use, especially once colonial structures were dismantled.

\noindent The variable \texttt{iron} shows a positive relationship with plantation size (\texttt{0.110}), indicating that countries with greater iron resources tend to have larger plantations. This could be explained by the fact that plantation economies often require significant infrastructure development, which may have been facilitated by access to local resources such as iron.

\noindent The variable \texttt{landlock} has a negative coefficient (\texttt{-0.358}), implying that landlocked countries are less likely to have large plantation economies. This could be because access to international trade routes is essential for the success of plantation economies, which are heavily reliant on export markets.

\noindent Similarly, the variable \texttt{Asia} shows a negative coefficient (\texttt{-0.103}), suggesting that countries in Asia tend to have smaller plantation economies compared to African regions. This difference may reflect distinct historical and institutional contexts, such as agrarian systems that did not rely on large-scale plantations.

\noindent Latitude, which affects climate conditions, plays a significant role in agricultural practices. The negative coefficient for latitude (\texttt{-1.907}) indicates that tropical regions (closer to the equator) were more likely to have large plantations suited for cash crops, given their favorable climate conditions.

\noindent Finally, the variable \texttt{deaths\_per\_pop} shows a positive and statistically significant coefficient (\texttt{0.033}), meaning that high mortality rates, possibly due to diseases like malaria or yellow fever, may have been linked to a reliance on plantation economies. Fewer local workers could have pushed colonizers to rely on enslaved labor or imported workers to sustain plantation systems.

\vspace{0.2 in} 
\noindent In the second-stage regression, the results reveal a significant and positive relationship between plantation economies and modern GDP, even after controlling for potential endogeneity. This result is consistent with the literature on extractive colonial institutions (Acemoglu et al., 2001; Nunn, 2008), which argue that plantation economies were designed to extract resources in a way that privileged the colonizers, thereby creating long-term economic structures that persist even after colonization ends. The coefficient for the plantation variable implies that colonies with large landholding plantation economies—where landowners were detached from both wage labor and forced-labor colonial subjects—experienced a 0.719 increase in modern GDP, relative to colonies without such plantations. This finding suggests that plantation-based economic structures contributed to persistent economic advantages, possibly through institutional and infrastructural channels that facilitated capital accumulation, productivity growth, or the entrenchment of economic elites.

\noindent The coefficient for \texttt{COLYEARS} is positive and statistically significant (\texttt{0.003}) in the second stage, suggesting that longer colonization periods are associated with higher levels of economic development in the post-colonial era, contrary to the negative relationship observed in the first stage. This shift may reflect the importance of long-term institutional investments made during colonization that, in some cases, promoted modern development.

\noindent The variable \texttt{iron} becomes negatively significant in the second stage (\texttt{-0.101}), suggesting that when controlling for plantation economies, iron resources may have less of a direct positive impact on economic outcomes than in the first stage. This could indicate that plantation economies themselves were a more important driver of modern economic development than the availability of iron.

\noindent The variable \texttt{landlock} also shows a significant negative effect (\texttt{-0.077}) in the second stage, further emphasizing the importance of geographical factors, such as trade access, in shaping economic development. Geographic isolation, particularly in terms of access to international trade routes, seems to hinder the development of plantation economies, which are heavily export-dependent.

\noindent The positive relationship with \texttt{latitude} (\texttt{5.083}) suggests that countries at higher latitudes may have had more favorable long-term economic conditions after the colonial period. This could reflect the fact that temperate regions, as opposed to tropical ones, often developed more diversified economies with better infrastructure and institutions after decolonization.

\noindent The variable \texttt{immigr foreign w} (immigration of foreign workers) shows a statistically significant positive effect on economic outcomes, with a coefficient of 0.417. This result suggests that countries with higher levels of immigration of foreign workers tend to experience better economic performance.
Many colonial powers relied on importing labor for plantation economies, which shaped the demographic composition and labor market structure of these countries (Acemoglu et al., 2001). After decolonization, the continued migration of foreign workers could have contributed to economic development by providing cheap labor for industries such as agriculture and manufacturing.

\noindent Finally, the \texttt{deaths\_per\_pop} variable shows a negative relationship with the outcome variable, though the result is not statistically significant. High mortality rates during and after colonial rule could have hindered economic development due to the loss of human capital or the diversion of resources to health-related efforts.

\vspace{0.3 in} 

\noindent The second model (M2: Without IV) in Table 2 presents a regression where the instrumental variable (IV) is excluded. This omission implies that the model does not account for endogeneity, which can lead to biased and inconsistent estimates. Specifically, by excluding the IV, the model fails to address potential correlations between the endogenous variable and unobserved factors that may influence both the dependent and independent variables.

\noindent In the second stage of Table 2, the coefficient for plantation economy is 0.295, which is lower than the estimate of 0.719 in the main model. This discrepancy suggests that the inclusion of the IV in the main model leads to a larger estimated effect of plantation economy on the outcome variable, likely economic development. Without the IV, the model underestimates this effect, as it does not correct for the endogeneity problem.

\noindent Similarly, the coefficient for iron is -0.071 in the second model, compared to -0.101 in the main model. This reduction further indicates that excluding the IV diminishes the estimated relationship between iron and the outcome variable.

\noindent Other variables, such as landlock, Asia, and French colonial influence, also exhibit different magnitudes when compared to the main model. This variation underscores the importance of the IV in obtaining more reliable and consistent estimates.

\noindent In conclusion, the exclusion of the instrumental variable in the second model results in biased coefficient estimates. 

\vspace{0.3 in} 

\noindent In the third model (M3: Without Fixed Effects), the removal of fixed effects leads to a loss of precision in the estimates, which impacts the interpretation of key relationships. The coefficient on plantation size remains large (0.464) but shows increased standard errors (0.105 compared to 0.033 in the main model). This suggests that omitting fixed effects introduces omitted variable bias, reducing the precision of the plantation size estimate.

\noindent Although the explanatory power of the model remains strong (R² = 0.734), the reduced precision indicates that unobserved heterogeneity, factors that vary across regions but are not directly included in the model, may be driving some of the observed effects. The coefficient for plantation economy (0.643) remains large and positive but loses statistical significance. This implies that, while the relationship between plantation economy and the outcome variable is still economically meaningful, the removal of fixed effects introduces additional sources of variation, making it harder to identify a clear causal effect.

\noindent The impact of landlock (-0.120, insignificant) and iron deposits (-0.090, insignificant) further suggests that regional factors, which are controlled for through fixed effects, play a significant role in shaping long-term economic outcomes. Fixed effects account for unobserved heterogeneity at the regional or country level, capturing historical, institutional, and geographical factors that influence economic development. Without these controls, the model becomes less able to isolate the true relationships between the variables.

\vspace{0.3 in} 

\noindent In the fourth model (M4), the key modification is the inclusion of rainfall as an instrumental variable (IV) for plantation economies. The theoretical rationale for using rainfall as an IV is based on the assumption that its effect on economic outcomes operates solely through its influence on plantation economies. In this model, the coefficient for plantation economies increases significantly to 3.196, compared to 0.719 in the main model (M1). However, this coefficient is accompanied by a substantially larger standard error of 3.98 compared to the main model's standard error of 0.209. This increase in standard error suggests that the estimation of the plantation economy coefficient is much more uncertain in the IV model. The IV approach is inherently more sensitive to the validity of the instrument, and if the instrument (rainfall) is not perfectly valid, the results may suffer from increased imprecision. While the IV method is intended to correct for endogeneity, the larger standard error indicates that the precision of the estimate has been reduced, highlighting the trade-off between bias correction and estimation accuracy.

\vspace{0.3 in} 

\noindent Model 5 serves as a robustness check by introducing an additional factor—labor market conditions in 1988—that could potentially influence the results. Since the main findings of the Main Model (especially the significance of plantation size) hold up even with this additional variable, the conclusion that the plantation economy had a long-term effect on economic development is strengthened. This suggests that the results are not spurious or driven by omitted variables related to the labor market, thus providing a more robust foundation for the Main Model's conclusions.

\noindent The coefficient for the plantation economy in Model 5 is \(0.460^*\), which is highly significant (\(p < 0.01\)). This indicates that countries with a history of plantation economies (likely due to their colonial economic structures) have higher output per worker. The positive coefficient aligns with the idea that certain legacies from colonial periods, such as the plantation systems, could have long-lasting effects on economic productivity.

\noindent The negative sign for COLYEARS suggests that the longer a country was colonized, the lower its output per worker in 1988. This could be attributed to the long-term economic and institutional impacts of colonialism, which might have hindered the country's economic development in the post-colonial period.

\noindent The coefficient for iron availability is \(0.110^*\), indicating that countries with greater iron resources had higher output per worker in 1988. Iron is a crucial input in industrialization, so this result is consistent with the theory that access to important natural resources can drive economic productivity.

\noindent The negative coefficient for being landlocked (\(-0.358\)) suggests that countries without access to the sea face significant economic disadvantages. This is in line with the understanding that landlocked countries incur higher transportation costs and face barriers to trade, which can hinder economic performance.

\noindent On average, countries in Asia have slightly lower output per worker compared to other regions, after controlling for other factors. The positive and significant coefficient for French colonial legacy (\(0.283^*\)) implies that countries that were former French colonies tend to have higher output per worker. 

\noindent The results from Model 5 are consistent with the theoretical understanding that colonial legacies, geography, and natural resources play a significant role in shaping long-term economic outcomes. The robustness of the coefficients across models, with similar signs and statistical significance, further strengthens the validity of these findings.

\newpage

\section{Conclusion}

\noindent This study examines the long-term economic consequences of plantation economies, focusing on how colonial institutions shaped contemporary economic outcomes. The empirical analysis, using both instrumental variable (IV) approaches and various model specifications, provides robust evidence of the enduring impact of plantation economies on modern economic productivity.

\noindent The first-stage regression validates the instrumental variable (plantation size), which is strongly correlated with the endogenous regressor, plantation economies. The positive and statistically significant coefficient for plantation size confirms that larger plantation economies are associated with greater plantation sizes, allowing us to capture exogenous variation in plantation economies. This establishes plantation size as a reliable instrument for isolating the effects of plantation economies on modern economic outcomes.

\noindent In the second-stage regression, we find a significant and positive relationship between plantation economies and modern GDP. This suggests that countries with large plantation economies during the colonial period experienced long-lasting economic advantages. The positive effect remains even when controlling for other factors, such as colonization years, geographical characteristics, and resource availability.

\noindent However, the exclusion of the instrumental variable leads to a lower estimate for the coefficient of plantation economies, highlighting the importance of addressing endogeneity. This discrepancy indicates that failing to account for the potential bias due to endogeneity underestimates the true effect of plantation economies. The results underscore the necessity of using valid instruments to obtain consistent and reliable estimates.

\noindent Model 3, which omits fixed effects, demonstrates the critical role of controlling for unobserved regional factors. While the coefficient for plantation economies remains positive, its statistical significance diminishes, and standard errors increase. This loss of precision emphasizes that fixed effects account for important heterogeneity across regions that influences long-term economic outcomes. Without these controls, the model becomes less able to isolate the true impact of plantation economies, suggesting that regional factors—such as historical, institutional, and geographical differences—are essential in understanding economic development.

\noindent In Model 4, we introduced rainfall as an alternative instrument for plantation economies. The resulting coefficient for plantation economies was substantially higher than in the main model, but this increase came with significantly larger standard errors. This suggests that while rainfall is a valid instrument, its use introduces greater uncertainty in the estimates, underscoring the trade-off between correcting for endogeneity and the precision of the results.

\noindent Finally, Model 5 introduced labor market conditions in 1988 as an additional control. The coefficient for plantation economies remained robust and highly significant, further strengthening the argument that plantation economies have a lasting effect on economic productivity. This result holds even after controlling for factors such as iron resource availability and geographical constraints, such as being landlocked. The consistency of the coefficient across different model specifications indicates that the relationship between plantation economies and economic outcomes is not driven by omitted variables but rather reflects a genuine long-term effect.

\noindent In conclusion, the results from all models provide compelling evidence that plantation economies played a critical role in shaping modern economic outcomes. The use of instrumental variables, fixed effects, and additional controls ensures the reliability and robustness of the findings. The study demonstrates that the legacies of colonial plantation systems continue to influence economic productivity today, highlighting the lasting impact of colonial institutions on contemporary development.

\newpage

\section{References}
\begin{thebibliography}{99}

\bibitem{acemoglu2004} 
Acemoglu, Daron, Simon Johnson, and James Robinson. 2004. “Institutions as the Fundamental Cause of Long-Run Growth.” \textit{W},.

\bibitem{acemoglu2001} 
Acemoglu, Daron, Simon Johnson, and James A. Robinson. 2001. “The Colonial Origins of Comparative Development: An Empirical Investigation.” 

\bibitem{alawattage2009} 
Alawattage, Chandana, and Danture Wickramasinghe. 2009. “Institutionalisation of Control and Accounting for Bonded Labour in Colonial Plantations: A Historical Analysis.”

\bibitem{alden1976} 
Alden, Dauril. 1976. “The Significance of Cacao Production in the Amazon Region during the Late Colonial Period: An Essay in Comparative Economic History.”

\bibitem{correlates2016} 
Correlates of War. 2016. “Correlates of War – The Correlates of War Project.”

\bibitem{gordon2001} 
Gordon, Alec. 2001. “Contract Labour in Rubber Plantations: Impact of Smallholders in Colonial South-East Asia.” \textit{Economic and Political Weekly}.

\bibitem{lange2006} 
Lange, Matthew, James Mahoney, and Matthias vom Hau. 2006. “Colonialism and Development: A Comparative Analysis of Spanish and British Colonies.”

\bibitem{assenova2017} 
Assenova, Valentina A., and Matthew Regele. 2017. “Revisiting the Effect of Colonial Institutions on Comparative Economic Development.” \textit{PLoS ONE} 12 (5): e0177100. 

\bibitem{ziltener2017} 
Ziltener, Patrick, Daniel Künzler, and André Walter. 2017. “Measuring the Impacts of Colonialism: A New Data Set for the Countries of Africa and Asia.”

\bibitem{albouy2017} 
Albouy, David Y. 2017. “The Colonial Origins of Comparative Development: An Empirical Investigation: Comment.”

\bibitem{nunn2008} 
Nunn, Nathan. 2008. “Understanding the Long-Run Effects of Africa’s Slave Trades.” \textit{The Quarterly Journal of Economics} 123 (1): 139–176. 

\bibitem{engerman2000} 
Engerman, Stanley L., and Kenneth L. Sokoloff. 2000. “Factor Endowments, Inequality, and Paths of Development Among New World Economies.” \textit{Economics and Politics} 12 (3): 235–276. 

\bibitem{austin2008} 
Austin, Gareth. 2008. “African Economic Development: Growth, Reversals, and Deep Transitions.” \textit{Journal of African Economies} 17 (5): 771–798. 

\bibitem{frankema2013} 
Frankema, E.H.P. 2013. \textit{The Cambridge Economic History of the Modern World}. Cambridge University Press.

\end{thebibliography}


\end{document}
